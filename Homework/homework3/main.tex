\documentclass{article}
\usepackage[utf8]{inputenc}

\title{CSE 15: Homework 3}
\author{Jatnael Montes}
\date{February 26, 2020}

\begin{document}

\maketitle

\section*{Knight and Knaves}
\begin{enumerate}
\item One day a traveller was wandering around the island of Knights and Knaves, when he encountered
two local inhabitants, P and Q. The traveller asked: “Is any of you a knave?”. P replied: “At least one of us is a knave”.
Can you find out what P and Q are? If so, what are they? If not, explain why not, and what other 

\emph{Answer: P is the knight, while Q is knave.}

\item Later on, the traveller met two other locals, A and B. He asked whether either of them is a knight.
A replied: “If B is a knave, then I am a knave too”.
What are A and B?

\emph{Answer: A is a Knight and B is also a Knight}
\end{enumerate}
\section*{Logical Identities}
\indent Simplify the following propositions. Show all steps of your solutions.\\

\begin{enumerate}
\item
$\neg(p \to (q \to p))$ \\
\begin{itemize}
    \item $\neg(\neg p \land (q \to p))$\\
    \item $p \land \neg (q \to p))$\\
    \item $p \land \neg (\neg q \lor p)$\\
    \item $p \land (q \land \neg p)$\\
    \item $p \land \neg p \land q$\\
    \item $p \land \neg p = False $\\
\end{itemize}
\emph{Answer: $False \land q = False $\\}

\item
$\neg((p \land q) \to (q \lor p))$\\
\begin{itemize}
    \item$\neg(\neg (p \land q) \lor (q \lor p))$\\
    \item$(p \land q) \land \neg (q \lor p)$\\
    \item$(p \land q) \land \neg q \land \neg p$\\
    \item$(p \land \neg p) \land \neg q \land \neg q$\\
\end{itemize}

\emph{Answer: $False = False$}

\end{enumerate}
\section*{Logical Equivalences}

Determine whether or not the following pairs of propositions are equivalent. Show all steps.\\
\begin{enumerate}
\item p $\to$ (q $\to$ r) and (p $\land$ q) $\to$ r

\begin{tabular}{|c|c|c|c|c|c|c|}
\hline
$q$ & $r$ & $p$ & $q \to r$ & $p \land q$ & $p \to (q \to r)$ & $(p \land q) \to r$ \\
\hline
F & F & F & T & F & T & T \\
F & F & T & T & F & T & T \\
F & T & F & T & F & T & T \\
F & T & T & T & F & T & T \\
T & F & F & F & F & T & T \\
T & F & T & F & T & F & F \\
T & T & F & T & F & T & T \\
T & T & T & T & T & T & T \\
\hline
\end{tabular}

\emph{Answer:The final two columns depicting the pair of propositions are equivalent to one another}

\item p $\to$ (q $\to$ r) and (p $\to$ q) $\to$ r \\
\begin{tabular}{|c|c|c|c|c|c|c|}
\hline
$q$ & $r$ & $p$ & $q \to r$ & $p \to q$ & $p \to (q \to r)$ & $(p \to q) \to r$ \\
\hline
F & F & F & T & T & T & F \\
F & F & T & T & F & T & T \\
F & T & F & T & T & T & T \\
F & T & T & T & F & T & T \\
T & F & F & F & T & T & F \\
T & F & T & F & T & F & F \\
T & T & F & T & T & T & T \\
T & T & T & T & T & T & T \\
\hline
\end{tabular}

\emph{Answer: The pair of propositions are not equivalent as the final two columns show.}
\end{enumerate}
\section*{Logical Consequence}
\begin{enumerate}
\item Jimmy is smart\\
\underline{Smart people are rich}\\
Jimmy is rich\\

\emph{Answer: It is impossible for the conclusion to be false if we assume the current conditions to be true, therefore this is a valid argument}

\item Islands are surrounded by water \\
\underline{Puerto Rico is surrounded by water}\\
Puerto Rico is an island\\

\emph{Answer:It is not valid to label Puerto Rico as an island solely because it is surrounded by water, since being an island is a subset of being surrounded by water.}
\end{enumerate}


\end{document}
