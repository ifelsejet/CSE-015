\documentclass{article}
\usepackage[utf8]{inputenc}
\usepackage{amsmath}

\title{Homework 4}
\author{Jatnael Montes }
\date{March 8, 2020}

\begin{document}

\maketitle

\section*{Exercises}
Prove (or disprove) the following results, showing all steps of your argument.
\begin{enumerate}
\item The sum of two odd integers is even. \\
Let a = 2n+1, b = 2k+1 
\begin{align*}
a + b = (2n + 1) + (2k + 1) \\
= 2n + 2k + 2\\
= 2(n + k + 1)\\
\end{align*}
Therefore the result holds



\item The sum of two even integers is even. \\
Let a = 2n, b = 2k 
\begin{align*}
a + b = 2n + 2k \\
= 2(n + k)
\end{align*}
Therefore the result holds

\item The square of an even number is even. \\
Let a = 2n
\begin{align*}
a^2 = (2n)^2 \\
=4n^2 \\
= 2(2n)^2 \\
\end{align*}
Therefore the result holds

\item The product of two odd integers is odd. \\
Let a = 2n+1, b = 2k+1 
\begin{align*}
ab = (2n + 1)(2k + 1) \\
= 4kn + 2n + 2k + 1 \\
= 2(2kn + n + k) + 1 \\
\end{align*}
Therefore the result holds

\item If $n^3$ + 5 is odd then n is even, for any n $\in$ Z. \\
We can solve this by using contraposition, so let n = 2k+1
\begin{align*}
n^3 + 5 = (2k + 1)^3 + 5 \\
= 8k^3 + 12k^2 + 6k + 1 + 5\\
= 8k^3 + 12k^2 + 6k + 6 \\
= 2(4k^3 + 6k^2 + 3k + 3) \\
\end{align*}
Therefore the original result holds

\item If 3n + 2 is even then n is even, for any n $\in$ Z. \\
We can solve this using contraposition, so let n = 2k+1
\begin{align*}
3n + 2 = 3(2k + 1) + 2 \\
= 6k + 3 + 2 \\
= 6k + 4 + 1 \\
= 2(3k + 2) + 1 \\
\end{align*}
Therefore the original result holds

\item The sum of a rational number and an irrational number is irrational. \\
We can prove this via contradiction, Let I be a irrational number and let n = $p/q$

\begin{align*}
n + I =a/b \\
p/q + I = a/b \\
I =a/b - p/q \\
= (aq-bp)/bq \\
\end{align*}

This means that I is rational, a contradiction to our original statement

\item The product of two irrational numbers is irrational. \\
Since we know that $\sqrt{2}$ is irrational, we can do the following:

\begin{align*}
\sqrt{2}\sqrt{2} = 2^{\frac{1}{2}} 2^{\frac{1}{2}} \\
= 2^1 \\
= 2 \\
\end{align*}
So a product of two irrationals is rational, therefore result is false.

\item Use mathematical induction to prove that the first n even integers add up to n(n + 1) \\
\textbf{Base case: }Set n = 1
\begin{align*}
2 = 1(1 + 1) = 2 \\
\end{align*}
Therefore the base case holds 

\textbf{Induction Hypothesis: }Assume the property holds when n = $k - 1$ \\
\begin{align*}
     2 + 4 +. . .+ 2(k-1) =k(k-1) \\
\end{align*}
\textbf{Inductive Step: }Show that if property holds for k-1 then it must hold for k \\
\begin{align*}
    2+4+...+2(k-1)=k(k-1)  \\
    = k^2-k+2k \\
    = k^2+k \\
    = k(k+1)
\end{align*}
Therefore the property holds for all positive integers.


\item Use mathematical induction to prove that $n^3$ + 2n is divisible by 3. \\
\textbf{Base case:} Set n = 1
\begin{align*}
1^3 + 2(1)/3 = 3/3 = 1 \\
\end{align*}
 Therefore the base case holds \\
 
 \textbf{Induction Hypothesis:} Assume if it works for n = k then $k^3$ + 2k is divisible by 3
 
 \textbf{Inductive Step:} Show that if it works for k, then it will work for k+1
 \begin{align*}
 (k+1)^3 + 2(k+1) \\
 = k^3+3k^2+3k+1+2k+2 \\
 = k^3+2k+3k^2+3k+3 \\
 = k^3+2k+3(k^2+k+1) \\
 \end{align*}
 Therefore result is true for all n.
 

\end{enumerate}

\end{document}
